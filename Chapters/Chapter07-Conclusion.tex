\chapter{Conclusion}
\label{cha:conclusion}
\epigraph{``Tut, tut, child!'' said the Duchess. ``Everything’s got a
moral,\\if only you can find it.''}{\textit{Alice’s Adventures in Wonderland}\\\textsc{Lewis Carroll}}

This, then, is the grand summary of what you have accomplished.  You may
well imagine that many readers will read \autoref{cha:introduction}, and
then skip right to \autoref{cha:conclusion}, and if, and only if, those two
parts are interesting, might be tempted to read the rest. A consequence is
that you should ensure that the reader will gain a good overall
understanding of what you have done by reading only the conclusion.  Thus,
this is a place to summarise all that has gone before, before finally
concluding on the results of your experiments and the validity of your
hypotheses. It is also important to ensure that \autoref{cha:introduction}
(which in all likelihood was written first) still aligns closely with the
work done and the conclusions reached.

Apart from reporting on your results, \ie your product, this is also a good
place to reflect on your process, and to discuss the wider consequences or
ramifications of your work. What would you like to impart to others going
down your path of inquiry? What would you have liked to have known, when you
first began your work?

If you so desire, this is also where you might add a section on Future Work,
where you point in the directions that should be followed to complete the
work you have already accomplished. This should, however, probably be the
very last thing on your to-do list.


%%% Local Variables:
%%% mode: latex
%%% TeX-master: "../ClassicThesis"
%%% ispell-dictionary: "british" ***
%%% fill-column: 76 ***
%%% End:
