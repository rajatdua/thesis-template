%*******************************************************
% Foreword
%*******************************************************

\chapter*{Some Thoughts on Tooling}
\label{cha:some-thoughts-tool}


As can be gleaned from the very existence of this guide, I very much favour PDF\LaTeX\ as the best way to format a thesis.
Once it has been properly setup and configured, it is unparalleled in consistent quality of output.
While excellent online editors exist, notably Overleaf\footnote{\url{https://www.overleaf.com/} accessed 2019/3/27.}, I would hesitate to recommend their use for a whole thesis, as I find that dedicated text editors, such as GNU Emacs, Vim, Sublime Text, or Visual Studio Code are vastly superior.
They are mature text editing platforms, and provide excellent support, not only for \LaTeX\ itself, but also for versioning, and thus for collaboration.

If you prefer a more visual tool, there are specialised \LaTeX\ editors, such as  \mLyX\footnote{\url{https://www.lyx.org/} accessed 2018/6/27.}, which is available for Linux, Windows, and macOS; \TeX nicCenter\footnote{\url{http://www.texniccenter.org/} accessed 2018/6/27.} for Windows, which this very document comes prepared for; and \TeX studio\footnote{\url{http://texstudio.sourceforge.net} accessed 2018/7/3.}, available for Linux, Windows, and macOS.

My own setup has for decades consisted of GNU Emacs with the packages AUC\TeX\footnote{\url{https://www.gnu.org/software/auctex/} accessed 2018/6/27.} combined with Ref\TeX\footnote{\url{https://www.gnu.org/software/auctex/reftex.html} accessed 2018/6/27.} (both easily installed from inside Emacs: Options $\rightarrow$ Manage Emacs Packages) and I have yet to see anything getting close to the power of that combination.

\section*{Installing \LaTeX}
\label{sec:installing-latex}

Installing and maintaining \LaTeX\ used to be a bit daunting, but is today quite straightforward using \TeX\ Live\footnote{\url{https://tug.org/texlive/} accessed 2018/6/27.} for Windows; Mac\negthinspace\TeX\footnote{\url{https://tug.org/mactex/} accessed 2018/6/27.} for mac\-OS; and your favourite package manager for your flavour of Linux (search for `\texttt{texlive}').

\section*{Handling Bibliographies}
\label{sec:handl-bibl}

\mBibTeX\ is indispensable when it comes to handling references.
I have included a starting set of references with this guide, but you will obviously need to add your own as your work progresses.  That is very much aided by tools---I use the excellent Bibdesk\footnote{\url{https://bibdesk.sourceforge.net/} accessed   2018/6/27.} (part of the Mac\negthinspace\TeX\ distribution mentioned above) on macOS to handle my bibliographies, and a Windows alternative could be the Mendeley\footnote{\url{https://blog.mendeley.com/2011/10/25/howto-use-mendeley-to-create-citations-using-latex-and-bibtex/}   accessed 2018/6/27.}  desktop client, which can export to \mBibTeX.
If you find yourself with references in another format than the one you need, \texttt{bibutils}\footnote{\url{https://sourceforge.net/p/bibutils/home/Bibutils/}   accessed 2018/7/4.} may come in handy.

Once you have your references, proper tools, such as Ref\TeX\ mentioned above, make inserting references a breeze.
The best place to find references in general is Google Scholar\footnote{\url{https://scholar.google.dk/} accessed 2018/6/27.}, which, just as most of the sites referenced by it, can export to \mBibTeX.
As you create your entries, you should take care to ensure that all fields required for others to find the referenced work are filled out correctly.
See \autoref{sec:small-note-refer} for a more detailed discussion on this.

While Google Scholar is fine for finding references, you will occasionally need to retrieve papers from ACM, IEEE, or some other paywalled site.
The AU Library has accounts with these and many other publishers, so if you access the sites from within AU (or through the AU VPN), you can usually download without restriction.

\section*{Creating Figures}
\label{sec:creating-figures}

PDF\LaTeX\ can include figures in many formats, notably \acs{PDF}, \acs{PNG}, and \acs{JPG}, but it is also possible to create quite sophisticated native \LaTeX\ figures using, \eg Tikz\footnote{\url{http://www.texample.net/tikz/examples/area/computer-science/} accessed 2018/6/27.}. See \autoref{sec:some-notes-tables} for more details.

\section*{Getting Help}
\label{sec:getting-help}
The organisation of the files comprising this template is explained in \autoref{sec:organization}.
A good starting point for getting help would be the \LaTeX\ wiki books\footnote{\url{https://en.wikibooks.org/wiki/LaTeX} accessed 2018/6/27.}, which does an admirable job of covering material for beginners and advanced users
alike.
\url{https://tex.stackexchange.com/} is an excellent resource for tricky \LaTeX\ related questions.
If you prefer your reference material in hard copy, the two seminal works are \citetitle{Lamport1994:LADPSUGARM1994}~\cite{Lamport1994:LADPSUGARM1994} by \citeauthor{Lamport1994:LADPSUGARM1994}, the creator of \LaTeX,
and \citetitle{Mittelbach2004:TLC2004}~\cite{Mittelbach2004:TLC2004} by \citeauthor{Mittelbach2004:TLC2004}.

\section*{This Document is Duplex}
\label{sec:print-this-docum}

It may seem obvious, but just to be clear: this document style is intended to be printed duplex.


%%% Local Variables:
%%% mode: latex
%%% eval: (visual-line-mode t)
%%% ispell-dictionary: "british" ***
%%% mode: flyspell
%%% TeX-master: "../ClassicThesis"
%%% End:
