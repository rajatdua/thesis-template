%*******************************************************
% Foreword
%*******************************************************

\chapter*{Some Thoughts on Tooling}
\label{cha:some-thoughts-tool}


As can be gleaned from the very existence of this guide, I very much
favour PDF\LaTeX\ as the best way to format a thesis. Once it has been
properly setup and configured, it is unparalleled in consistent
quality of output.  While excellent online editors exist, notably
Share\LaTeX\footnote{\url{https://www.sharelatex.com/}} and
Overleaf\footnote{\url{https://www.overleaf.com/}}, I would hesitate
to recommend their use for a whole thesis, as I find that dedicated
text editors, such as GNU Emacs, Vim, Sublime Text, or Visual Studio
Code are vastly superior.  They are mature text editing platforms, and
provide excellent support, not only for \LaTeX\ itself, but also for
versioning, and thus for collaboration.

If you prefer a more visual tool, there are specialised \LaTeX\
editors, such as LyX\footnote{\url{https://www.lyx.org/}}, which is
available for Linux, Windows, and macOS, and \TeX
nicCenter\footnote{\url{http://www.texniccenter.org/}} for Windows,
which this very document comes prepared for.

My own setup has for decades consisted of GNU Emacs with the packages
AUC\TeX\footnote{\url{https://www.gnu.org/software/auctex/}} combined
with
Ref\TeX\footnote{\url{https://www.gnu.org/software/auctex/reftex.html}}
(both easily installed from inside Emacs: Options $\rightarrow$ Manage
Emacs Packages) and I have yet to see anything getting close to the
power of that combination.

\section*{Installing \LaTeX}
\label{sec:installing-latex}

Installing and maintaining \LaTeX\ used to be a bit daunting, but is today
quite straightforward using \TeX\
Live\footnote{\url{https://tug.org/texlive/}} for Windows,
Mac\TeX\footnote{\url{https://tug.org/mactex/}} for mac\-OS, and your
favourite package manager for your flavour of Linux.

\section*{Handling Bibliographies}
\label{sec:handl-bibl}

\textsc{Bib}\negthinspace\TeX\ is indispensable when it comes to
handling references. I have included a starting set of references with
this guide, but you will obviously need to add your own as your work
progresses.  That is very much aided by tools---I use the excellent
Bibdesk\footnote{\url{https://bibdesk.sourceforge.net/}} (part of the
Mac\TeX\ distribution mentioned above) on macOS to handle my
bibliographies, and a Windows alternative could be the
Mendeley\footnote{\url{https://blog.mendeley.com/2011/10/25/howto-use-mendeley-to-create-citations-using-latex-and-bibtex/}}
desktop client, which can export to \textsc{Bib}\negthinspace\TeX.
Once you have your references, proper tools, such as Ref\TeX\
mentioned above, make inserting references a breeze.  The best place
to find references in general is Google
Scholar\footnote{\url{https://scholar.google.dk/}}, which, just as
most of the sites referenced by it, can export to
\textsc{Bib}\negthinspace\TeX. As you create your entries, you should
take care to ensure that all fields required for others to find the
referenced work are filled out correctly.

\section*{Creating Figures}
\label{sec:creating-figures}

PDF\LaTeX\ can include figures in many formats, notably \acs{PDF},
\acs{PNG}, and \acs{JPG}, but it is also possible to create quite
sophisticated native \LaTeX\ figures using, \eg
Tikz\footnote{\url{http://www.texample.net/tikz/examples/area/computer-science/}}.

\section*{Getting Help}
\label{sec:getting-help}

An good starting point would be the \LaTeX\ wiki
books\footnote{\url{https://en.wikibooks.org/wiki/LaTeX}}, which does
an admirable job of covering material for beginners and advanced users
alike.  \url{https://tex.stackexchange.com/} is an excellent resource
for tricky \LaTeX\ related questions. If you prefer your reference
material in hard copy, the two seminal works are \LaTeX: A Document
Preparation System: User's Guide And Reference
Manual~\cite{Lamport1994:LADPSUGARM1994} by Leslie Lamport, the creator of \LaTeX,
and The \LaTeX\ Companion~\cite{Mittelbach2004:TLC2004}
by Mittelbach \etal.

\section*{This Document is Duplex}
\label{sec:print-this-docum}

It may seem obvious, but just to be clear: this document style is intended to be printed duplex.


%%% Local Variables:
%%% mode: latex
%%% TeX-master: "../ClassicThesis"
%%% End:
