\chapter{Related Work}
\label{cha:related-work}

Whereas the purpose of \autoref{cha:introduction} was to entice and
convince the reader that work reported is interesting, that the author
is asking the right questions about it, and reading about it will be
worthwhile, the purpose of \autoref{cha:related-work} is to
demonstrate that the author possesses a fine overview and keen
understanding of the topic of the work.  Note that while the title of
the chapter is ``Related Work'', it might as well be called
\emph{``Relevant} Work'' in that you should only include work that are
directly useful or relevant to your purpose. 

Writing about others' work can be challenging---it is easy to succumb
to just writing condensed summaries, which are just as tedious to read
as they are to write. A better method is to gain an overview over the
field of inquiry, and then establish in the first section what aspects
or dimensions are crucial to systems or methodologies such as the ones
described. This demonstrates to the reader that the author has
understanding and judgement. Having done this, every paper or work can
then be described in those established terms. This makes for easier
and much more structured writing, and it also helps the reader
differentiate the systems and works reported on. If there are multiple
works that cover approximately the same area (\eg using the same
technique), you may mention several, but only go into detail with the
most significant or representative one.

The chapter can then be concluded with a table summarising all the
work reported on using the aspects defined in the introduction of the
chapter.

A crucial element of this chapter is that it concerns the work of
others and \emph{only} that. While the selection of aspects or
dimensions described above invariantly will reflect your own focus,
that should be the extend of which your own work and plans influence
this chapter.  Your own judgement comes in the next chapter.

\section{Frameworks and Technologies}
\label{sec:fram-techn}

Related work need not be only published academic work. In many cases,
it is also relevant to describe crucial frameworks and technologies
that will be used or are relevant for the thesis.  This does not mean
that all employed technologies should be described in detail, but
frameworks and technologies that are unusual (for lack of a better
word) could be described here. \Eg there is no need to describe an
ordinary network stack, but if the work involves GPU programming, a
description of the chosen architecture might well be relevant, as it
informs all the following chapters.

%%% Local Variables:
%%% mode: latex
%%% TeX-master: "../ClassicThesis"
%%% End:
