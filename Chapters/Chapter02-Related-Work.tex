\chapter{Related Work}
\label{cha:related-work}

Whereas the purpose of \autoref{cha:introduction} was to entice and
convince the reader that work reported is interesting, that the author
is asking the right questions about it, and reading about it will be
worthwhile, the purpose of \autoref{cha:related-work} is to
demonstrate that the author possesses a fine overview and keen
understanding of the topic of the work.  Note that while the title of
the chapter is ``Related Work'', it might as well be called
\emph{``Relevant} Work'' in that you should only include work that are
directly useful or relevant to your purpose. 

Writing about others' work can be challenging---it is easy to succumb
to just writing condensed summaries, which are just as tedious to read
as they are to write. A better method is to gain an overview over the
field of inquiry, and then establish in the first section the aspects
or dimensions that are crucial to systems or methodologies such as the
ones described. This demonstrates to the reader that the author has
understanding and judgement. Having done this, every paper or work can
then be described in those established terms. This makes for easier
and much more structured writing, and it also helps the reader
differentiate the systems and works reported on. If there are multiple
works that cover approximately the same area (\eg using the same
technique), you may mention several, but only go into detail with the
most significant or representative one.

The chapter can then be concluded with a table summarising all the
work reported on using the aspects defined in the introduction of the
chapter.

A crucial element of this chapter is that it concerns the work of
others and \emph{only} that. While the selection of aspects or
dimensions described above invariantly will reflect your own focus,
that should be the extend of which your own work and plans influence
this chapter.  Your own judgement comes in the next chapter.


\section{A small note on references}
\label{sec:small-note-refer}

It is essential for any scientific work to cite its sources. This can be done
relatively painlessly using \textsc{Bib}\negthinspace\TeX\ as outlined on
\autopageref{sec:handl-bibl}. It is important that you include all necessary
information in the bibliography for others to correctly identify and locate
the referenced work. If you cite one work, it should be done thus
\cite{Kristensen2010:MP2P2010}, if you cite a specific page in a work, it
should be done thus \cite[p. 410]{Chawathe2003:2003}, and if you cite multiple
works, it should be done thus
\cite{knuth:1976,knuth:1974,Kristensen2010:MP2P2010,Mittelbach2004:TLC2004}.  I generally 
reserve the bibliography for works that have been properly published and/or
peer reviewed. References to Web pages are best handled through
footnotes\footnote{\url{https://www.tug.org/applications/pdftex/}}, though
some things, such as RFCs and other standard documents, belong in the proper
bibliography. If you use a figure from another work, you \emph{must} give
attribution---otherwise it counts as plagiarism, which is a serious offence. 


\section{Frameworks and Technologies}
\label{sec:fram-techn}

Related work need not be only published academic work. In many cases,
it is also relevant to describe crucial frameworks and technologies
that will be used or are relevant for the thesis.  This does not mean
that all employed technologies should be described in detail, but
frameworks and technologies that are unusual (for lack of a better
word) could be described here. \Eg there is no need to describe an
ordinary network stack, but if the work involves GPU programming, a
description of the chosen architecture might well be relevant, as it
informs all the following chapters.

\section{Summary}
\label{sec:summary}
As described in \autoref{cha:related-work}, ending the chapter with a figure
like \autoref{tab:relatedwork-summary} summarising the findings can be a great
way to remind the reader of the results, as well as laying the foundation for
\autoref{cha:analysis}.


\begin{table}[h]
    \myfloatalign
  \begin{tabularx}{\textwidth}{Xccccc} \toprule
    \tableheadline{System} & \tableheadline{Aspect} & \tableheadline{Aspect} & \tableheadline{Aspect} & \tableheadline{Aspect} & \tableheadline{Aspect} \\ \midrule
    Foo & ++ & +  & -- & +  & --\\
    Bar & +  & ++ & -- & --  & +\\
    Baz & +  & --  & -- & ++ & ++\\
   Quux & --  & +  & + & +  & ++\\
    \bottomrule
  \end{tabularx}
  \caption[Summary of systems]{The systems and papers described in \autoref{cha:related-work}. The systems have been rated from low to high along the chosen aspects.}
  \label{tab:relatedwork-summary}
\end{table}


%%% Local Variables:
%%% mode: latex
%%% TeX-master: "../ClassicThesis"
%%% ispell-dictionary: "british" ***
%%% mode:flyspell ***
%%% mode:auto-fill ***
%%% fill-column: 78 ***
%%% End:
