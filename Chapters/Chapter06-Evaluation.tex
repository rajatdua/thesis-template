\chapter{Evaluation}
\label{cha:evaluation}

Having built the equivalent of a experimental setup, it is time to use
the implementation to test the hypotheses.

This is usually broken down in stages and subquestions.

A structured approach to performing and reporting on experiments is
to follow this pattern for every single experiment:

\begin{enumerate}
\item What is the purpose of the experiment?
\item What is the expected outcome?
\item What are the parameters under which the experiment takes place?
\item What are the results?
\item How do the results align with the expected outcome? If they do not
  align, why is that so?
\end{enumerate}

Results should be presented summarised. For quantitative experiments, this
will usually be in the form of tables and graphs.  Remember to note the
number of times experiments were repeated, as well as averages, and standard
deviations (in percent of the mean).  There is much more to the proper
evaluation of experimental data than can be expounded upon here, but I turn
the reader's attention to \citep{Downey2011:TSPASFP2011}, which is freely
available.

If your results are of a qualitative nature, the summaries will depend on
the type of investigation you have done. It can be carefully annotated
recordings of specific incidents of the system in use; analysis with quotes
from interviews; or results from questionnaires and other investigations.
Whole transcripts belong in an appendix, but excerpts are fine in the main
text.

\section{Some Notes on Tables, Graphs, and Figures}
\label{sec:some-notes-tables}

It is paramount that you clearly and concisely present your results, and
that usually entails documentation in the form of tables or figures.  Such
things are more informative if they are clearly formatted and legible.

There are many tools with which you can plot graphs of your data. I would
suggest that you use something that can be scripted, so you can tweak your
parameters, run the evaluation, and see the fresh results, again and
again. Having to, \eg copy and paste or import data into an application will
slow you down, and might well lead to confusion---is this the new data or
not? A small investment upfront in scripting everything will lead to
savings in both time and frustration later on.

If you are fluent in Python,
Matplotlib\footnote{\url{http://matplotlib.org/}} is very good, and can be
used interactively in
Jupyter\footnote{\url{http://jupyter.org/}}. Gnuplot\footnote{\url{http://gnuplot.info/}}
is another powerful choice, and probably easier to get started with. If you
know R\footnote{\url{https://www.r-project.org/}},
ggplot2\footnote{\url{http://ggplot2.org/}} is the obvious route.  For
web-based systems, D3.js\footnote{\url{http://d3jsp.org/}} is quite popular.

If you output your data to comma or tab separated files, you can even
generate your plots and tables directly in \LaTeX\ using
\textsc{Pgfplots}\footnote{\url{http://pgfplots.sourceforge.net/}}
   (further examples\footnote{\url{http://pgfplots.net/tikz/examples/all/}}) and
\textsc{PgfplotsTable}\footnote{\url{http://pgfplots.sourceforge.net/pgfplotstable.pdf}}.
Doing this can quickly save you time, as especially \LaTeX\ tables, while
pretty, are tedious and error prone to edit. See
\autoref{tab:pretty-table} for an auto-generated example. For more help on
tables, see the \LaTeX\ wiki books\footnote{\url{https://en.wikibooks.org/wiki/LaTeX/Tables}}.


\begin{table}
  \myfloatalign
  \pgfplotstabletypeset[
  every head row/.style={
    before row=\toprule,
    after row=\midrule,
  },
  every even row/.style={
    before row={\rowcolor[gray]{0.95}}},
  every last row/.style={
    after row=\bottomrule
  },
  columns={x, avgy, stddevpp},
  columns/x/.style={
    column name=x,
    dec sep align,
    fixed,
    fixed zerofill,
    precision=2
  },
  columns/avgy/.style={
    column name=$\bar{y}$,
    dec sep align,
    fixed,
    fixed zerofill,
    precision=2
  },
  columns/stddevpp/.style={
    column name=$\sigma\%$,
    dec sep align,
    fixed,
    fixed zerofill,
    precision=1
  },
  ]{./data/example.csv}
  % \tableheadline{labitur bonorum pri no} & \tableheadline{que vista}
  % & \tableheadline{human} \\ \midrule
  \caption{An auto-generated table} \label{tab:pretty-table}
\end{table}


\begin{figure}
  \centering
  \begin{tikzpicture}
    \begin{axis} [xlabel=$x$, ylabel=$\bar{y}$]
      \addplot[
      error bars,
      y dir=both,
      y explicit,
      error bar style={red}
      ] table[
      x=x,
      y=avgy,
      y error=stddevp]{./data/example.csv};
    \end{axis}
  \end{tikzpicture}
  \caption{A graph with error bars}
  \label{fig:pretty-graph}
\end{figure}


%%% Local Variables:
%%% mode: latex
%%% TeX-master: "../ClassicThesis"
%%% ispell-dictionary: "british" ***
%%% mode:flyspell ***
%%% mode:auto-fill ***
%%% fill-column: 76 ***
%%% End:
